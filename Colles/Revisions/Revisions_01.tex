\documentclass[10pt,fleqn]{article} % Default font size and left-justified equations
\usepackage[%
    pdftitle={Exercices de SII},
    pdfauthor={Xavier Pessoles}]{hyperref}
\usepackage{afterpage}
\newcommand\blankpage{%
    \null
    \thispagestyle{empty}%
    \addtocounter{page}{-1}%
    \newpage}
    
\newcommand{\repRel}{../../..}
\input{\repRel/Style/packages}
\input{\repRel/Style/new_style}
\input{\repRel/Style/macros_SII}
\input{\repRel/Style/environment}
\usepackage{\repRel/Style/UPSTI_Pedagogique}

\newcommand{\macrocomp}{macro_competences}
\newcommand{\comp}{competences}
\newcommand{\td}{fichier_td}
\newcommand{\repExo}{dossier}
\newcommand{\repStyle}{\repRel/Style}

\def\xxYCartouche{-2.25cm}
\def\xxYongletGarde{.5cm}
\def\xxYOnget{.9cm}

\begin{document}

\def\xxcompetences{}
\def\xxfigures{}

\graphicspath{{\repStyle/png/}{\repRel/PSI_Cy_01_ModelisationSystemes/Ch_01_Generalites/Cours/images/}}

%\setlength{\columnseprule}{.1pt}

%% MODELISER
% Page de garde

\livrettrue
\input{\repStyle/Revisions_Entete}


\normaltrue
\newpage
\proffalse \colletrue





%%%% Paramétrage du TD %%%%
\def\xxactivite{ Révisions} % \normalsize \vspace{-.4cm}
\def\xxauteur{\textsl{Xavier Pessoles}}


\def\xxnumchapitre{}%Chapitre 1 \vspace{.2cm}}
\def\xxchapitre{}%\hspace{.12cm} Approche énergétique}

\def\xxcompetences{}%
%\vspace{-.5cm}
%\footnotesize{
%\textsl{%
%\textbf{Savoirs et compétences :}\\
%\vspace{-.2cm}
%\begin{itemize}[label=\ding{112},font=\color{ocre}] 
%\item Mod2.C18.SF1 : Déterminer l’énergie cinétique d’un solide, ou d’un ensemble de solides, dans son mouvement par rapport à un autre solide.
%\item Res1.C1.SF1 : Proposer une démarche permettant la détermination de la loi de mouvement.
%%\item Mod1.C5.SF2 : Déterminer la puissance des actions mécaniques extérieures à un solide ou à un ensemble de solides, dans son mouvement rapport à un autre solide.
%%\item Mod1.C5.SF3 : Déterminer la puissance des actions mécaniques intérieures à un ensemble de solides.
%\end{itemize}}}}

\def\xxtitreexo{Trains épicycloïdaux \& Lois de Coulomb}
\def\xxsourceexo{}%\hspace{.2cm} \footnotesize{X -- ENS -- PSI -- 2012}}

\def\xxfigures{}%
%\includegraphics[width=.57\linewidth]{fig_00}
%}%figues de la page de garde


\input{\repRel/Style/pagegarde_TD}
\setcounter{numques}{0}

\setlength{\columnseprule}{.1pt}

\pagestyle{fancy}
\thispagestyle{plain}


\vspace{5.2cm}

\def\columnseprulecolor{\color{ocre}}
\setlength{\columnseprule}{0.4pt} 

%%%%%%%%%%%%%%%%%%%%%%%

\setcounter{exo}{0}

\ifprof
%\begin{multicols}{2}
\else
\begin{multicols}{2}
\fi

\renewcommand{\repExo}{28_TrainEpi}
\graphicspath{{\repStyle/png/}{\repRel/ExercicesCompetences/C2_MettreEnOeuvreDemarche/C2_06_Transmetteurs/\repExo/images/}}
\input{\repRel/ExercicesCompetences/C2_MettreEnOeuvreDemarche/C2_06_Transmetteurs/\repExo/\repExo.tex}

\renewcommand{\repExo}{29_TrainEpi}
\graphicspath{{\repStyle/png/}{\repRel/ExercicesCompetences/C2_MettreEnOeuvreDemarche/C2_06_Transmetteurs/\repExo/images/}}
\input{\repRel/ExercicesCompetences/C2_MettreEnOeuvreDemarche/C2_06_Transmetteurs/\repExo/\repExo.tex}

\renewcommand{\repExo}{30_TrainEpi}
\graphicspath{{\repStyle/png/}{\repRel/ExercicesCompetences/C2_MettreEnOeuvreDemarche/C2_06_Transmetteurs/\repExo/images/}}
\input{\repRel/ExercicesCompetences/C2_MettreEnOeuvreDemarche/C2_06_Transmetteurs/\repExo/\repExo.tex}


\renewcommand{\repExo}{532_MAM_Frottement_Cylindre}
\graphicspath{{\repStyle/png/}{\repRel/ExercicesCompetences/B2_ProposerModele/B2_14_ModeliserAction_Frottement/\repExo/images/}}
\input{\repRel/ExercicesCompetences/B2_ProposerModele/B2_14_ModeliserAction_Frottement/\repExo/\repExo.tex}

\renewcommand{\repExo}{533_MAM_Frottement_Cylindre}
\graphicspath{{\repStyle/png/}{\repRel/ExercicesCompetences/B2_ProposerModele/B2_14_ModeliserAction_Frottement/\repExo/images/}}
\input{\repRel/ExercicesCompetences/B2_ProposerModele/B2_14_ModeliserAction_Frottement/\repExo/\repExo.tex}



\ifprof
%\end{multicols}%
\else
\end{multicols}%
\fi







\end{document}



